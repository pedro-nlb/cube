\documentclass[notheorems, hyperref]{beamer}

%% KEY LINES IN THIS TEX FILE %% (enter line number+gg to go)

%
% LOCAL FONT DEFINITIONS -- need to come first
%
%\usepackage{mathpazo}
%\usepackage{libertine}
%\usepackage[libertine]{newtxmath}
%\usefonttheme[onlymath]{serif}

%
% STANDARD PREAMBLE
%
\input{preamble}
\allowdisplaybreaks

%
% ABOUT FONT DEFINTIONS IN THE PREAMBLE
%
% Mathscr for sheaves use \sA, where A can be any letter. Exceptions and additions:
% % \E (vector bundles)
% % \F (coherent sheaves)
% % \G (coherent sheaves)
% % \hom (sheaf hom)
% % \I (ideal sheaves)
% % \L (line bundles)
% % \M (line bundles)
% % \O (structure sheaf)
% % \w (canonical sheaf)
%
% Mathcal use \calA. Exceptions and additions:
% % \U (open cover)
% % \X (families of varieties)
% % \Y (families of varieties)
%
% Mathbb use \bbA. Exceptions and additions:
% % \A (affine space)
% % \C (complex numbers)
% % \Gm (puctured affine line)
% % \k (field)
% % \N (natural numbers)
% % \P (projective space)
% % \Q (rational numbers)
% % \R (real numbers)
% % \V (geometric vector bundle)
% % \Z (integers)
%
% Boldfont for categories use \bfA. Additions:
% % \Cat (categories)
% % \Coh (coherent sheaves)
% % \D (derived category)
% % \Db (bounded derived category)
% % \K (homotopy category)
% % \Mod (modules)
% % \PSh (presheaves)
% % \QCoh (quasi-coherent sheaves)
% % \Set (sets)
% % \Sh (sheaves)
% % \Top (topological spaces)
% % \Vec (vector bundles)
%
% Mathfrak for ideals
% % From \a to \e
% % \m and \n for maximal ideals

%
% THEOREM ENVIRONMENTS
%
% Theorems, propositions, etc (dark green)
\theoremstyle{darkgreentheorem}
\newtheorem{thm}{Theorem}
\newtheorem{reform}{Reformulation}
\newtheorem{lm}[thm]{Lemma}
\newtheorem{prop}[thm]{Proposition}
\newtheorem{cor}[thm]{Corollary}
\newtheorem{conj}[thm]{Conjecture}
% Definitions (dark blue)
\theoremstyle{darkbluedefinition}
\newtheorem{defn}[thm]{Definition}
% Examples (dark red)
\theoremstyle{darkredexample}
\newtheorem{exa}[thm]{Example}
% Remarks (black)
\theoremstyle{remark}
\newtheorem{rem}[thm]{Remark}
\newtheorem{nota}[thm]{Notation}
\newtheorem{fact}[thm]{Fact}
\newtheorem{q}[thm]{Question}
\newtheorem{pbl}[thm]{Problem}

%
% THEOREM CROSS-REFERENCING
%
\crefname{thm}{theorem}{theorems}
\Crefname{thm}{Theorem}{Theorems}
\crefname{lm}{lemma}{lemmas}
\Crefname{lm}{Lemma}{Lemmas}
\crefname{prop}{proposition}{propositions}
\Crefname{prop}{Proposition}{Propositions}
\crefname{cor}{corollary}{corollaries}
\Crefname{cor}{Corollary}{Corollaries}
\crefname{conj}{conjecture}{conjectures}
\Crefname{conj}{Conjecture}{Conjectures}
\crefname{defn}{definition}{definitions}
\Crefname{defn}{Definition}{Definitions}
\crefname{exa}{example}{examples}
\Crefname{exa}{Example}{Examples}
\crefname{rem}{remark}{remarks}
\Crefname{rem}{Remark}{Remarks}
\crefname{nota}{notation}{notations}
\Crefname{nota}{Notation}{Notations}
\crefname{fact}{fact}{facts}
\Crefname{fact}{Fact}{Facts}
\crefname{q}{question}{questions}
\Crefname{q}{Question}{Questions}
\crefname{pbl}{problem}{problems}
\Crefname{pbl}{Problem}{Problems}

%
% MATH OPERATORS
%
\DeclareMathOperator{\Hom}{Hom}
\DeclareMathOperator{\Pic}{Pic}

%
% OTHER COMMANDS
%
\newcommand{\ot}{\otimes}
\newcommand{\op}{\oplus}
\renewcommand{\L}{\mathcal{L}}
\renewcommand{\M}{\mathcal{M}}

%
% TITLE PAGE INFORMATION
%
\title[Seesaw Principle and Theorem of the Cube]{Seesaw Principle and Theorem of the Cube}
\author{Remarks on Section I.5 of Milne's \textit{Abelian Varieties}}
\institute{University of Freiburg}
\date{26th May 2020}
 
%
% LINKS AND PDF OPTIONS
%
\makeatletter
\hypersetup{
  %pdfauthor={\authors},
  pdftitle={\@title},
  %pdfsubject={\@subjclass},
  %pdfkeywords={\@keywords},
  %pdfstartview={Fit},
  %pdfpagelayout={TwoColumnRight},
  %pdfpagemode={UseOutlines},
  bookmarks,
  colorlinks,
  linkcolor=linkblue,
  citecolor=linkred,
  urlcolor=linkred}
\makeatother
\usecolortheme{rose}
 
\begin{document}
 
\frame{\titlepage}
 
\begin{frame}
    \frametitle{Seesaw Principle --- Idea}
    \begin{thm}[Weil]
	A limit of trivial line bundles on a complete variety is again trivial.
    \end{thm}
    \pause
    \[ \downarrow \quad \quad \quad \text{\small{making this idea more precise}} \]

    \begin{thm}[Statement for varieties over $k=\bar{k}$]
	$V$ complete, $T$ any (``parameter space''), $\L\in \Pic(V\times T)$.
	Then
	\begin{center}
	    \begin{tikzcd}[ampersand replacement=\&]
		Z:=\{ t\in T\mid \L_{t} \text{ is trivial }\}\arrow[closed]{r} \& T
	    \end{tikzcd}
	\end{center}
	is closed and $\L|_{V\times Z}$ is the pullback of a line bundle on $Z$.
    \end{thm}
    \pause
    This is the conclusion of the subsection on the seesaw principle in \cite{mil08}, which can be read independently.
    This statement will be used to \textbf{reduce} the Theorem of the Cube to an easier case.
\end{frame}

\begin{frame}
    \frametitle{Seesaw Principle --- Generalisation}
    \begin{thm}[{\cite[Theorem 6.3]{bha17} or \cite[Theorem 5]{fra18}}]
	Let $f\colon X\to S$ be a proper flat morphism of (locally) noetherian schemes with geometrically integral fibres and let $\L\in X$.
	Then
	\begin{center}
	    \begin{tikzcd}[ampersand replacement=\&]
		Z:=\{s\in S\mid \L_{s} \text{ is trivial }\}\arrow[closed]{r} \& S
	    \end{tikzcd}
	\end{center}
	and $\L|_{f^{-1}(Z_{\mathrm{red}})}=f^{*}\M_{0}$ for some $\M_{0}\in \Pic(Z_{\mathrm{red}})$.
	Moreover, there exists a (unique) closed subscheme structure on $Z$ such that $\L|_{f^{-1}(Z)}=f^{*}\M$ for some $\M\in \Pic(Z)$ and $Z$ is universal amongst all (locally) noetherian $S$-schemes with this property.
    \end{thm}
    \pause
    Brian Conrad proves this using existence and separatedness of $\Pic_{X/S}$ \cite[Theorem 3.1.1]{con15}: $\O_{X}$ gives us $[\O_{X}]\colon S\to \Pic_{X/S}$, whose image we identify with $S$, and then $Z=[\L](S)\cap [\O_{X}](S)$.
    Conversely, assuming $\Pic_{X/S}$ exists, the seesaw theorem implies that it is separated \cite[Remark 2.4.2]{fra18} (automatic if $S=k$).
\end{frame}

\begin{frame}
    \frametitle{Seesaw Principle --- Particular case in a picture}
    \begin{center}
	\begin{tikzcd}[ampersand replacement=\&]
	    X \& \& Y \\
	    \& X\times Y\arrow{ul}\arrow[dashed]{dl}\arrow[dashed]{ur}\arrow{dr} \& \\
	    X \& \& Y \\
	    \& \&
	\end{tikzcd}
    \end{center}
    \pause
    We have some $\L\in \Pic(X\times Y)$.
    We can think of $\L$ as a weight distribution on the two extremes $X$ and $Y$ of this seesaw: the more points $x\in X$ such that $\L|_{\{ x\}\times Y}$ is trivial, the heavier $\L$ sits on $X$.
    \pause

    In this case, let's say, we have $\L|_{X\times \{y\}}$ trivial for all $y\in Y$.
    Then $\L$ sits with full weight on $Y$, so the seesaw principle tells us that there must be some line bundle on $Y$ from which $\L$ is the pullback.
\end{frame}

\begin{frame}
    \frametitle{Theorem of the Cube --- Statement}
    \begin{thm}[Statement for varieties over $k=\bar{k}$]
	$X,Y$ complete, $Z$ any and $\L\in \Pic(X\times Y\times Z)$ such that
	\[ \L|_{X\times Y\times \{z_{0}\}}, \L|_{X\times \{y_{0}\}\times Z} \text{ and }\L|_{\{x_{0}\}\times Y\times Z} \]
	are trivial.
	Then $\L$ is trivial.
    \end{thm}
    \begin{center}
	\begin{tikzpicture}
	    \pgfmathsetmacro{\cubex}{2}
	    \pgfmathsetmacro{\cubey}{2}
	    \pgfmathsetmacro{\cubez}{2}
	    \draw[black,fill=yellow] (0,0,0) -- ++(-\cubex,0,0) -- ++(0,-\cubey,0) -- ++(\cubex,0,0) -- cycle;
	    \draw[black,fill=yellow] (0,0,0) -- ++(0,0,-\cubez) -- ++(0,-\cubey,0) -- ++(0,0,\cubez) -- cycle;
	    \draw[black,fill=yellow] (0,0,0) -- ++(-\cubex,0,0) -- ++(0,0,-\cubez) -- ++(\cubex,0,0) -- cycle;
	    \draw(-1.5,-1.5,-1.5) node{XY};
	    \draw(-0.6,0.3,0) node{YZ};
	    \draw(0.4,-0.6,0) node{XZ};
	\end{tikzpicture}
    \end{center}
    If the restriction of $\L\in \Pic(\text{Cube})$ to each face $XY$, $YZ$ and $XZ$ is trivial, then $\L$ is trivial.
\end{frame}

\begin{frame}
    \frametitle{Theorem of the Cube --- Generalisation}
    As with the Seesaw principle, it is possible to generalise the Theorem of the Cube to the relative setting, see for example:
    \begin{enumerate}[label=\textbullet]
	\item \cite[Corollary 6.8]{bha17} for proper flat schemes with geometrically integral fibres over a noetherian base.
	\item \cite[Theorem 9]{fra18} for an abelian scheme over a locally noetherian base.
    \end{enumerate}
\end{frame}

\begin{frame}
    \frametitle{Theorem of the Cube --- An aplication}
    One application deserving its own name is \cite[Theorem I.5.5]{mil08}:
    \begin{thm}[Theorem of the Square for varieties over $k=\bar{k}$]
	Let $A$ be an abelian variety and $\L\in \Pic(A)$.
	For all $a\in A$ denote $t_{a}\colon A\to A$ the translation by $a$.
	Then for all $a,b\in A$ we have
	\[ \L\ot t^{*}_{a+b}\L \cong t^{*}_{a}\L\ot t^{*}_{b}\L. \]
    \end{thm}
    \pause

    \marginnote{\dbend} In the relative setting, LHS and RHS will differ by the pullback of a line bundle on the base scheme, so if $\Pic(S)=0$ we still have such an isomorphism but it is not canonical.
    On the other hand, the isomorphisms in \cite[Corollaries I.5.2--I.5.4]{mil08} are canonical even in the relative setting, cf.~\cite[Corollaries to Theorem 9]{fra18}.
\end{frame}

\begin{frame}
    \frametitle{References}
    \bibliographystyle{alpha}
    \bibliography{main.bib}
\end{frame}

\end{document}
